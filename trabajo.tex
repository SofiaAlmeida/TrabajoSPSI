\documentclass[11pt]{article}
\usepackage[spanish]{babel}
\usepackage[utf8]{inputenc}

%Tipo de letra y color
\usepackage{FiraSans}                 % Fuente sans serif
\usepackage[T1]{fontenc}
\usepackage[usenames]{color}
\definecolor{ColorTitulo}{RGB}{169,50,38}

%Imágenes
\usepackage{graphicx}
\graphicspath{{img/}}
\usepackage{float}

%Matemáticas
\usepackage{amssymb}
\usepackage{amsmath}

%Código
\usepackage{listings}
\usepackage{xcolor}
\renewcommand{\lstlistingname}{Código}% Listing -> Código

\definecolor{dkgreen}{rgb}{0,0.6,0}
\definecolor{gray}{rgb}{0.5,0.5,0.5}
\definecolor{mauve}{rgb}{0.58,0,0.82}

%% \lstset{frame=tb,
%%   language=Sage,
%%   aboveskip=3mm,
%%   belowskip=3mm,
%%   showstringspaces=false,
%%   columns=flexible,
%%   basicstyle={\small\ttfamily},
%%   numbers=none,
%%   numberstyle=\tiny\color{gray},
%%   keywordstyle=\color{blue},
%%   commentstyle=\color{dkgreen},
%%   stringstyle=\color{mauve},
%%   breaklines=true,
%%   breakatwhitespace=true,
%%   tabsize=3
%% }

\definecolor{codegreen}{rgb}{0,0.6,0}
\definecolor{codegray}{rgb}{0.5,0.5,0.5}
\definecolor{codepurple}{rgb}{0.58,0,0.82}
\definecolor{backcolour}{rgb}{0.95,0.95,0.92}
 
\lstdefinestyle{mystyle}{
    backgroundcolor=\color{backcolour},   
    commentstyle=\color{codegreen},
    keywordstyle=\color{magenta},
    numberstyle=\tiny\color{codegray},
    stringstyle=\color{codepurple},
    basicstyle=\ttfamily\footnotesize,
    breakatwhitespace=false,         
    breaklines=true,                 
    captionpos=b,                    
    keepspaces=true,                 
    numbers=left,                    
    numbersep=5pt,                  
    showspaces=false,                
    showstringspaces=false,
    showtabs=false,                  
    tabsize=2
}
\lstset{style=mystyle}

%Márgenes
\usepackage[paper=portrait, pagesize]{typearea}
\usepackage{titlepic}
\setlength{\textwidth}{155mm}
\setlength{\textheight}{210mm}
\setlength{\oddsidemargin}{6mm}
\setlength{\evensidemargin}{28mm}
\setlength{\topmargin}{-10mm}

%Símbolo para las listas
\usepackage{listings}
\usepackage{enumerate}
\renewcommand{\labelitemi}{$\circ$}
\usepackage{enumitem}

%Encabezados y pies de página
\usepackage{fancyhdr}
\pagestyle{fancy}
\fancyhf{}
\fancyfoot[R]{\thepage}
\lhead[L]{\MakeUppercase{\textit{Curva Elíptica}}}

\renewcommand{\sectionmark}[1]{
\markboth{#1}{}}
\renewcommand{\headrulewidth}{0pt} 

%Índice
\addtocontents{toc}{\hspace{-7.5mm} \textbf{Secciones}}
\addtocontents{toc}{\hfill \textbf{Página} \par}
\addtocontents{toc}{\vspace{-2mm} \hspace{-7.5mm} \hrule \par}

%Apéndices
\usepackage[toc,title,page]{appendix}

%Subíndices
\usepackage{amsmath}
\DeclareMathOperator{\Prob}{Prob}

\begin{document}

%%PORTADA%%
\begin{titlepage}
\centering
\vspace*{4.5cm}
\fontsize{30pt}{40pt}{\selectfont\sffamily{\textcolor{ColorTitulo}{Trabajo Teoría}}}\\
\vspace{5mm}
\fontsize{60pt}{50pt}{\selectfont\sffamily{\textcolor{ColorTitulo}{Curva Elíptica}}}
\vspace{1.5cm}

{\scshape\large \today \par}
\vspace{1cm}


\vspace*{\fill}
\raggedleft{Sofía Almeida Bruno\\
Pedro Manuel Flores Crespo\\
María Victoria Granados Pozo\\
\par}
\vspace*{-2cm}

\end{titlepage}

\thispagestyle{empty}
\tableofcontents

\newpage

\section{Introducción}

% Comentar que usaremos Sage en algún momento

\section{Curva Elíptica}
% Escribo esta sección inspirándome en el capítulo 2 de ``Elliptic Curves Number theory and cryptography'' y comparándo con otros artículos
Antes de entrar en cómo usar las curvas elípticas en criptografía debemos definirlas y estudiar su estructura de grupo.

Una curva elíptica será el conjunto de puntos que verifiquen la ecuación:
\[y^2 = x^3 + ax + b,\]
% Lo de 4a^3+27b \neq 0 no aparece por todos lados...TODO: revisar qué implica exactamente y ver si nos interesa o no...
% vale...igual sí que aparecía más adelante
% Imponemos esa condición a los valores de $a$ y $b$ para evitar que aparezcan raíces múltiples ¿por qué? supongo que porque puden dar problemas sobre las cosas generales que vamos a decir(?) aunque si las estudias de forma independiente pueden tener interés (no en nuestro caso que vamos a hacer un estudio más general)
donde $a$ y $b$ son constantes. Esta ecuación es la \textbf{ecuación de Weierstrass} para curvas elípticas. Normalmente, $a, b, x, y$ toman valores en un cuerpo. Por ejemplo: los números reales $\mathbb{R}$, los números complejos $\mathbb{C}$, los números racionales $\mathbb{Q}$, un cuerpo finito $\mathbb{F}_p$ para un primo $p$ (este es el que se utiliza en criptografía habitualmente), un cuerpo finito $\mathbb{F}_q$ donde $q = p^k,$ $k \ge 0$, \dots

De forma general, consideramos un cuerpo $K$ y definimos la curva elíptica $E$ sobre él de la siguiente forma:
\[E = \{(x,y) \in K\times K : y^2 = x^3 + ax +b\} \cup \{\infty\}.
\]
El punto $\infty$ lo incluimos por definición y más adelante veremos su definición.

Podemos usar Sage para visualizar algunos ejemplos de curvas elípticas. En el Código \ref{cod:EC} observamos la forma genérica de definir en Sage curvas elípticas.

%TODO: arreglar colores
% TODO: probar con language=Python
\begin{lstlisting}[label={cod:EC}, caption={Curva elíptica en Sage}, morekeywords={sage}]
sage:  E = EllipticCurve(K, [a, b]); E
Elliptic Curve defined by y^2  = x^3 + x + 3 over a field K
\end{lstlisting}

El Código \ref{cod:ej1} es el necesario para crear la curva con ecuación de WeiErstrass $y^2=x^3-1$.

\begin{lstlisting}[label={cod:ej1}, caption={Curva elíptica $y^2=x^3-x$}, morekeywords={sage}]
sage: E =  EllipticCurve(RR, [-1, 0]);
      plot(E, (-4, 3), color=hue(0.6))
\end{lstlisting}

Podemos observar la salida obtenida en la Figura \ref{fig:ej1}.

% Cuando terminemos de escribir ya reajustamos tamaño y posición de las figuras para que no queden espacios en blanco
\begin{figure}[H]
    \centering
    \includegraphics[width=0.5\textwidth]{ej1}
    \caption{Salida Código \ref{cod:ej1}}
    \label{fig:ej1}
\end{figure}

Programamos de forma similar el código para visualizar la gráfica de la curva dada por la ecuación $y^2=x^3+x$, véase el Código \ref{cod:ej2}, cuya salida se encuentra en la Figura \ref{fig:ej2}

\begin{lstlisting}[label={cod:ej2}, caption={Curva elíptica $y^2=x^3+x$}, morekeywords={sage}]
sage: E =  EllipticCurve(RR, [1, 0]);
      plot(E, (-2, 4), color=hue(0.8))
\end{lstlisting}

\begin{figure}[H]
    \centering
    \includegraphics[width=0.5\textwidth]{ej2}
    \caption{Salida Código \ref{cod:ej2}}
    \label{fig:ej2}
\end{figure}

En la Figura \ref{fig:EllipticCurveCatalog} tenemos una tabla con la forma de las curvas elípticas para valores enteros de $a$ entre -2 y 1 y $b$ entre -1 y 2.

\begin{figure}[H]
    \centering
    \includegraphics[width=0.5\textwidth]{EllipticCurveCatalog}
    \caption{Ejemplos de curvas elípticas}
    \label{fig:EllipticCurveCatalog}
\end{figure}

%lo necesito para la parte de ECDSA sino se usa en otro sitio ponerlo en ECDSA

Parámetros de dominio para los algoritmos (p, a, b, G, n, h) donde:
\begin{itemize}
	\item p: primo
	\item a, b: Coeficientes de la curva elíptica, hay que elegirlos con cuidado para que el algoritmo sea seguro
	\item G: Generador del grupo
	\item n: Orden del grupo
	\item h: Cofactor del subgrupo
\end{itemize}

\subsection{El problema del logaritmo discreto}
La seguridad de la criptografía de la curva elíptica reside en la dificultad para calcular logaritmos discretos. Este problema es análogo el de otros criptosistemas como el Digital Signatura Algorithm (DSA) o el intercambio de llaves mediante el mecanismo de Diffie-Hellman.

En nuestro caso, el problema consiste en lo siguiente: ``si conocemos $ P $ y $ Q $ dos puntos de la curva elíptica, ?`podemos encontrar un $ k $ tal que $ Q = kP $? ''. Vemos que no estamos calculando un logaritmo como tal pero se sigue usando ese término por analogía a otros sistemas como se ha mencionado anteriormente. 

Actualmente, no existe ninguna demostración matemática que efectivamente pruebe la dificultad de dicho cálculo. Solo sabemos que es difícil pero no podemos estar seguros. Con el siguiente ejemplo ilustramos su complejidad. Para ello usamos un algoritmo ``Baby-step, giant-step'' que se basa en el hecho de que cualquier entero $ x $ puede expresar como $ x = am+b $ con $ a,m \text{ y } b$ también enteros. Escribimos entonces:
\begin{equation*}
\begin{split}
Q &= xP \\
  &= (am+b)P\\
  &=amP + bP.
\end{split}
\end{equation*}
Así, podemos expresar $ Q - amP  = bP $. El algoritmo consiste en en calcular algunos valores de $ bP $ y otros de $ Q-amP $ hasta que encontremos una correspondencia. El algoritmo también indica que debemos escoger $ m = \sqrt{n} $ y los números $ a $ y $ b $ se mueven entre 0 y $ m $ por lo que mientras $ bP $ tiene incrementos pequeños (``baby'') $ amP $ los tiene grandes (``huge''). Este método nos aporta una complejidad tanto en tiempo como en memoria de $ O(\sqrt{n}) $ . Este ataque tiene entonces complejidad exponencial pero es mejor que uno de fuerza bruta. De todos modos sigue siendo intratable ya que para un $  n  $ tal que $ \sqrt{n} = 7.922816251426434 \times 10^{28} $ el almacenamiento de datos que podemos llegar a necesitar es de $ 2,5\times10^{30} $ bytes de memoria (la capacidad de almacenamiento mundial es de aproximadamente $ 10^{21} $ bytes).

\section{Algoritmos}

\subsection{ECDSA}

ECDSA es un algoritmo de firma digital, Elliptic Curve Digital Signature Algorithm, es una variante del algoritmo DSA (Digital Signature Algortihm) aplicado a curvas elípticas. Trabaja con el hash del mensaje en lugar de con el propio mensaje. La elección de la función hash es importante, de esto dependerá la seguridad del sistema criptográfico. El hash del mensaje tendrá una longitud de \textit{n} bit.

Imaginemos que Alice quiere firmar un mensaje con su llave privada ($d_{A}$), y la otra persona, Bob, quiere validar la firma con la llave pública de Alice ($H_{A}$). Alice es la única que puede producir las firmas válidas, sin embargo todo el mundo que tenga su llave pública puede verificarlas.

Alice y Bob están usando los parámetros del dominio. El hash truncado lo denotaremos por \textit{z}.

Algoritmo de firma del mensaje de Alice, a partir de \textit{k} y \textit{z} se genera la firma con la clave privada de Alice:


\begin{enumerate}
	\item Tomamos un entero k de forma aleatorio en el conjunto \{ 1, ..., n-1\}.
	\item Calcular \textit{P} = \textit{kG}.
	\item Calcular el número \textit{r = $x_{P}$}.
	\item Si \textit{r} es 0 entonces se toma otro \textit{k} y se intenta de nuevo.
	\item Se calcula \textit{s} = \textit{$k^{-1}$}(\textit{z} + \textit{r$d_{A}$}) mod \textit{n}  con \textit{$k^{-1}$} el inverso multiplicativo de \textit{k} módulo \textit{n}.
	\item Si \textit{s} es 0 entonces se elige otro \textit{k} y se vuelve al principio.
\end{enumerate} 

Al final obtendremos la firma que será la pareja (\textit{r}, \textit{s}) 

Ahora entra el juego Bob que para validar la firma, a partir del mensaje firmado y de \textit{z} con la clave pública de Alice


\subsection{ECDH}

\subsection{Comparación con RSA}

\section{Conclusiones}




%Apéndice
\appendix
\renewcommand{\thesection}{\Roman{section}}
\setcounter{section}{0}
\section{Glosario}



\end{document}
